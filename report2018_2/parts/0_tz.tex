\begin{center}
Министерство науки и высшего образования Российской федерации\\
Федеральное государственное образовательное учреждение высшего образования\\
<<ТОМСКИЙ ГОСУДАРСТВЕННЫЙ УНИВЕРСИТЕТ СИСТЕМ УПРАВЛЕНИЯ И РАДИОЭЛЕКТРОНИКИ>> (ТУСУР)\\
Кафедра комплексной информационной безопасности электронно-вычислительных систем (КИБЭВС)\\

\itshape{Групповое проектное обучение}\\
\end{center}

\vfill

\begin{flushright}
\begin{minipage}{0.45\textwidth}
 \begin{flushleft}
УТВЕРЖДАЮ\\
Заведующий каф. КИБЭВС\\
\underline{\hspace{2.5cm}} А.А. Шелупанов\\
<<\underline{\hspace{1cm}}>>\underline{\hspace{3cm}} 2018г.\\
 \end{flushleft}
\end{minipage}
\end{flushright}

\vfill

\begin{center}
ТЕХНИЧЕСКОЕ ЗАДАНИЕ\\
на выполнение инновационного проекта № КИБЭВС-1808\\
\end{center}

\vfill

\begin{enumerate}
\item Основание для выполнения проекта: приказ № 208ст от 24.01.2018\par
\item Наименование проекта: Разработка и развитие системы для обучения и проведения практик по информационной безопасности\par
\item Цель проекта: Разработать платформу для проведения практик по информационной безопасности в формате тестирования на проникновение внутри среды, моделирующей работу информационной инфраструктуры организации\par
\item Основные задачи проекта на этапах реализации: реализация развертывания сервиса внутри контейнера, настройки подсети — порта для контейнера и импортирования/экспортирования задач и настроек сервиса, исправление ряда существующих ошибок, доработка и улучшение системы, добавление возможности оценки квестов, разработка документации по системе\par
\item Научная новизна проекта: Генерация компьютерной сети для каждого участника (либо команды) и устройств в сети, выполняющих какую-либо работу и обменивающихся информацией друг с другом. Идея состоит в том, чтобы смоделировать реальные сети организаций и предоставить участнику доступ к ней либо возможность получения доступа к сети. После чего участнику предлагается попытка повысить свои привилегии в сети, получить доступ к устройствам, получить доступ к информации или получить контроль над сетью\par
\item Планируемый срок реализации: планируемый срок создания системы 2 года\par
\item Целевая аудитория (потребители): пользователи, заинтересованные в получении практических навыков в сфере информационной безопасности\par
\item Заинтересованные стороны:  Факультет безопасности\par
\item Источники финансирования и материального обеспечения: Разработка ведется за счет личных средств группы\par
\item Ожидаемый результат (полученный товар, услуга): В результате работы над данным проектом будет создана система, позволяющая создавать и эксплуатировать компьютерные сети с целью оттачивания навыков проведения тестов на проникновение\par
\item Руководитель проекта:  Никифоров Д.С., младший научный сотрудник кафедры КИБЭВС\par
\item Ответственный исполнитель проекта: Дудкин Данил Геннадьевич гр. 725\par
\item Члены проектной группы:\par
\hspace{2cm}Дудкин Данил Геннадьевич гр. 725\par
\hspace{2cm}Косенко Екатерина Игоревна гр. 716\par
\hspace{2cm}Михайлова Александра Андреевна гр. 725\par
\hspace{2cm}Ушев Сергей Дмитриевич гр. 725\par
\item Место выполнения проекта: ауд. 805\par
\item Календарный план выполнения проекта\par
\end{enumerate}
	
\vspace{\baselineskip}

\noindent Таблица 1 -- Состав и содержание работ по созданию (развитию) объекта разработки и вводу в эксплуатацию\\

\begin{tabular}{|p{0.5cm}|p{2cm}|p{3cm}|p{3cm}|p{6cm}|}
\hline № этапа & Наименование этапа & Содержание работы & Сроки выполнения & Ожидаемый результат этапа\\ \hline 
1 & Подготовительный & Развертывание тестового сервиса внутри контейнера Исправление ошибок и доработка системы Изучение набора компьютерной верстки \LaTeX & 04.10.2018 & Работоспособный тестовый сервис внутри контейнера Исправленные ошибки Навыки работы в \LaTeX\\
\hline

\end{tabular}

\vspace{\baselineskip}

\begin{flushleft}
\begin{minipage}{0.45\textwidth}
 \begin{flushleft}
Руководитель проекта:\\
\underline{\hspace{1.5cm}} Никифоров Д.С.
<<\underline{\hspace{1cm}}>>\underline{\hspace{3cm}} 20\underline{\hspace{0.5cm}}г.\\
 \end{flushleft}
 \end{minipage}
\end{flushleft}

\begin{flushright}
\begin{minipage}{0.45\textwidth}
\begin{flushright}
Члены проектной группы:\\
\underline{\hspace{1.5cm}} Дудкин Данил Геннадьевич гр. 725\\
\underline{\hspace{1.5cm}} Косенко Екатерина Игоревна гр. 716\\
\underline{\hspace{1.2cm}} Михайлова Александра Андреевна гр. 725\\
\underline{\hspace{1.5cm}} Ушев Сергей Дмитриевич гр. 725\\
 \end{flushright}
\end{minipage}
\end{flushright}
\clearpage
