\section*{Заключение}
\addcontentsline{toc}{section}{Заключение}
\pretolerance10000

По истечению сроков, отведенных на выполнение групповой проектной работы, большинство запланированных на семестр задач были выполнены. Проведена работа с контейнерами (реализована ассинхронная обработка операций для работы с контейнерами, развернуты сервисы внутри контейнера, настроен порт для контейнера), исправлены ошибки с подсчетом рейтинга участников и изменением имен пользователей, написан учебный материал и разработано практическое задание <<Smash The Stack -- IO>> для FreeHackQuest, изучена система компьютерной верстки документов \LaTeX, с использованием которой подготовлены следующие разделы документации по системе: общее представление о системе, структура системы, описание стека технологий и процессов системы.\par 
В следующем семестре планируется продолжить работу над разработкой платформы для проведения практик по информационной безопасности в формате тестирования на проникновение внутри среды, моделирующей работу информационной инфраструктуры организации.\par
Будущие результаты ГПО рекомендуется использовать при проведении практик по обучению тестированию на проникновение, а также при подготовке специалистов по информационной безопасности. Разрабатываемая система не только поможет устраивать практики по тестированию на проникновение, но и облегчит работу организаторов соревнований CTF, сделав процесс развертывания и настройки всех подсистем более простым и комфортным.\par
Также в ходе выполнения групповой проектной работы были освоены компетенции, необходимые для получаемой специальности, а именно:
\begin{itemize}
\pretolerance10000 
\item ОПК-8: способность к освоению новых образцов программных, технических средств и информационных технологий;
\item ПК-6: способность проводить анализ, предлагать и обосновывать выбор решений по обеспечению эффективного применения автоматизированных систем в сфере профессиональной деятельности;
\item ПК-24: способность\hfilобеспечить\hfilэффективное\hfilприменение \\информационно-технологических ресурсов автоматизированной системы с учетом требований информационной безопасности.\par
\end{itemize}
\vspace{\baselineskip}

Отчет по результатам группового проектного обучения выполнен в соответствии с образовательным стандартом ТУСУР [11].
\clearpage
