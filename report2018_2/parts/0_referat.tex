\begin{center}
РЕФЕРАТ
\end{center}
\pretolerance10000
\vspace{\baselineskip}

Групповая проектная работа, 42 страницы, 16 рисунков, 11 источников, 3 приложения.\par
ИНФОРМАЦИОННАЯ БЕЗОПАСНОСТЬ, FREEHACKQUEST, АВТОМАТИЗИРОВАННАЯ СИСТЕМА, КОНТЕЙНЕРИЗАЦИЯ, LXD, C++, LATEX, GNU DEBUGGER, PLANTUML\par
Объектом разработки является система для обучения и проведения практических занятий по информационной безопасности.\par
Цель работы -- разработка платформы для проведения практических занятий\hfilпо\hfilинформационной\hfilбезопасности\hfilв\hfilформате\hfilтестирования\hfilна\hfil\\проникновение внутри среды, моделирующей работу информационной инфраструктуры организации.\par
В текущем семестре была продолжена работа с контейнерами и в результате были реализованы асинхронные операции для работы с контейнерами, развернуты сервисы внутри контейнера, настроен порт для контейнера, а также исправлены ошибки с подсчетом рейтинга участников и изменением имен пользователей, сформирован учебный материал и разработано практическое задание <<Smash The Stack -- IO>> для FreeHackQuest, изучена система компьютерной верстки документов \LaTeX, с использованием которой подготовлены следующие разделы документации по системе: общее представление о системе, структура системы, описание стека технологий и процессов системы.\par
В качестве инструментария для выполнения данной работы были использованы: язык программирования C++ с библиотекой STL, \LaTeX, PlantUml, Gnu Debugger.\par
\clearpage
