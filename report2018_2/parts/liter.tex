\section*{Список использованных источников}
\addcontentsline{toc}{section}{Список использованных источников}
\pretolerance10000
\begin{enumerate}
\item[1 ]C++ Программирование -- Система CMake [Электронный ресурс]. -- Режим доступа: http://procplusplus.blogspot.ru/2011/06/cmake–1.html (дата обращения:29.11.2018).
\item[2 ]Tproger -- STL: стандартная библиотека шаблонов C++ [Электронный ресурс]. -- Режим доступа: https://tproger.ru/articles/stl–cpp/ (дата обращения: 29.11.2018).
\item[3 ]Linux Containers -- LXD introduction [Электронный ресурс]. -- Режим доступа: https://linuxcontainers.org/lxd/ (дата обращения: 29.11.2018).
\item[4 ]An introduction to LaTeX [Электронный ресурс]. -- Режим доступа: https://www.latex-project.org/about/ (дата обращения: 5.12.2018).
\item[5 ]PlantUML -- все, что нужно бизнес-аналитику для создания диаграмм в программной документации [Электронный ресурс]. -- Режим доступа: https://habr.com/post/416077/ (дата обращения: 5.12.2018).
\item[6 ]GDB: The GNU Project Debugger [Электронный ресурс]. -- Режим доступа: http://www.gnu.org/software/gdb/ (дата обращения: 6.12.2018).
\item[7 ] Разбор эксплойта уязвимости CVE-2015-7547 [Электронный ресурс]. -- Режим доступа: https://cyberleninka.ru/article/n/razbor-eksployta-uyazvimosti-cve-2015-7547 (дата обращения: 7.12.18).
\item[8 ]PT-2018-14: Переполнение буфера в PHOENIX CONTACT FL SWITCH [Электронный ресурс]. -- Режим доступа: https://www.securitylab.ru/lab/PT-2018-14 (дата обращения: 7.12.18).
\item[9 ] Smash the Stack IO Level 3 Writeup [Электронный ресурс]. -- Режим доступа: http://raidersec.blogspot.com/2012/10/smash-stack-io-level-3-riteup.html (дата обращения: 8.12.2018). 
\item[10 ] Smash The Stack Wargaming Network [Электронный ресурс]. -- Режим доступа: http://smashthestack.org/wargames.html (дата обращения: 9.12.2018).
\item[11 ]Чернышев А.А., Кормилин В.А. ОС ТУСУР 01-2013 Образовательный стандарт ВУЗа, 2013. -- 53 с.
\end{enumerate} 
\clearpage