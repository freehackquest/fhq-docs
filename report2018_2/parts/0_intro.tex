\section*{Введение}
\addcontentsline{toc}{section}{Введение}
\pretolerance10000

FreeHackQuest (FHQ) -- это платформа для обучения и проведения соревнований по компьютерной безопасности. FHQ включает в себя учебник, различные задачи для решения, связанные с администрированием, криптографией, компьютерно-криминалистической экспертизой, стеганографией и многими другими направлениями информационной безопасности.\par
Ключевая концепция проведения обучения и соревнований заключается в генерации компьютерной сети, состоящей из сервисов и устройств, выполняющих какую-либо работу и обменивающихся информацией друг с другом. Идея состоит в том, чтобы смоделировать реальные сети предполагаемых организаций и предоставить участникам соревнований доступ (или возможность получения доступа) к данной сети. Затем каждому обучающемуся или игроку предлагается получить доступ к устройствам и информации или контроль над сетью.\par
Актуальность нашего проекта определяется колоссальным ростом скорости развития информационных технологий и постоянной потребностью в практической подготовке специалистов в сфере информационной безопасности.\par
Целью данного проекта является разработка платформы для проведения практически\hfilзанятий\hfilпо\hfilинформационной\hfilбезопасности\hfilв\hfilформате\\тестирования на проникновение внутри среды, моделирующей работу информационной инфраструктуры организации.\par
\clearpage