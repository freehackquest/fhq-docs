\begin{center}
 LaTex
\end{center}

\vspace{\baselineskip}

LaTeX — это собирательное название для системы подготовки (верстки) документов. Она включает набор инструментов, которые из текстовых файлов, записанных с использованием специального языка разметки формируют готовые к печати документы (как правило в формате PDF). Собственно TeX — это низкоуровневый язык разметки и программирования который лежит в основе этой системы. \par
Особенности LaTex:
\begin{itemize}
\item удобное средство для написания технических отчетов и различной документации;
\item стиль, шрифты, оформление таблиц, рисунков и т.д. согласованы во всём документе;
\item большие документы можно разбивать на несколько файлов и работать с ними отдельно, в том числе с использованием систем управления версиями;
\item легко создаются алфавитные указатели, сноски, ссылки и библиографические списки;
\item удобно включать такие вставки, как исходный код, математические формулы.
\end{itemize}
\vspace{\baselineskip}

