\addcontentsline{toc}{section}{Приложение А (Справочное) Статья для Интернет-конференции ГПО}
\pretolerance1000

\begin{center}
Приложение А\\
(Справочное)\\
Статья для Интернет-конференции ГПО
\end{center}
\vspace{\baselineskip}

\begin{center}
\textbf{ИСПОЛЬЗОВАНИЕ ТЕХНОЛОГИИ КОНТЕЙНЕРИЗАЦИИ В СФЕРЕ ИНФОРМАЦИОННОЙ БЕЗОПАСНОСТИ}\\
\textbf{\textit{А.А. Михайлова, С.Д. Ушев, студенты каф. КИБЭВС}}\\
\textit{Научный руководитель: Д.С. Никифоров, мл.н.с. каф. КИБЭВС}\\
\textit{г. Томск, ТУСУР, 725\_maa@fb.tusur.ru}\\
\textbf{\textit{Проект ГПО КИБЭВС-1808 – Разработка и развитие системы для обучения и проведения практик по информационной безопасности}}
\end{center}
В данной статье рассматривается технология контейнеризации, ее применение в рамках текущего проекта, преимущества и недостатки.\par
\textbf{Ключевые слова: }виртуализация, контейнеризация, Linux-контейнер, LXD.\par
Основная идея проекта -- обучение участников практических занятий по информационной безопасности посредством участия в тестировании на проникновение имитированной инфраструктуры организации.\par
В рамках проекта было необходимо решить задачу запуска уязвимых сервисов и приложений,  с помощью которых участникам необходимо проникнуть в сеть организации и получить доступ к информации ограниченного доступа, которая в рамках практик называется флагом. В ходе практики участник будет искать уязвимости в имитированной инфраструктуре организации, помогающие получить контроль над одним из сервисов. Из этого вытекает риск, что участник сможет получить контроль не только над сервисами, но и над системой проведения практик по информационной безопасности. В связи с чем было решено использовать виртуализацию. Если участник и сможет получить контроль, то это будет виртуальная машина, а не реальная система.\par
Рассматривались такие типы виртуализации, как полная виртуализация, паравиртуализация и виртуализация на уровне операционной системы. В итоге была выбрана виртуализация на уровне операционной системы по причине ненужности виртуализации аппаратной платформы. Именно отсутствие расходов на эмуляцию оборудования и работы виртуального программного обеспечения на реальном оборудовании позволяет получить высокую производительность.\par
Технология контейнеризации -- это метод, позволяющий запускать приложение и необходимый ему минимум системных библиотек изолированно от других процессов [1]. Контейнеризация является виртуализацией на уровне операционной системы, то есть изоляция приложения осуществляется с помощью средств операционной системы.\par
Технология контейнеризации начала свое развитие и получила широкое применение в системах GNU/Linux, поэтому стоит обозначить термин Linux-контейнер или просто контейнер.\par
Linux-контейнер -- это набор процессов, изолированный от остальной операционной системы и запускаемый с отдельного образа операционной системы, который содержит все файлы, необходимые для их работы. Образ содержит все зависимости приложения и поэтому может легко переноситься из среды разработки в среду тестирования, а затем в промышленную среду [2].\par
Для запуска и управления контейнерами выбран системный менеджер Linux-контейнеров Linux Container Daemon (LXD).\par
К преимуществам использования контейнеров относятся:
\begin{itemize}
\item быстрый запуск: запускается как процесс, а не виртуальная машина;
\item высокая производительность и малые задержки;
\item эффективное использование памяти [3];
\item единая среда исполнения для разработки, тестирования и эксплуатации.
\end{itemize}
\vspace{\baselineskip}

Недостаток заключается в новых угрозах безопасности для хостовой операционной системы. Под хостовой операционной системой (хост) понимается система, на которой происходит выполнение контейнеров, а также запуск и управление виртуальными (гостевыми) операционными системами.\par
  Можно выделить следующие угрозы безопасности для хостовой системы. при использовании контейнеризации:
\begin{itemize}
\item одно общее ядро хостовой операционной системы для всех контейнеров, эксплуатация уязвимости которого в одном из контейнеров ведет к компрометации хоста;
\item увеличение поверхности атаки;
\item доступ к файловой системе хоста из контейнера;
\item запуск привилегированных контейнеров: существуют эксплоиты, позволяющие выходить из таких контейнеров и получать полный контроль над хостом;
\item некорректное разграничение прав доступа на хосте.
\end{itemize}
\vspace{\baselineskip}

\par Средства операционной системы с ядром Linux, позволяющие изолировать выполнение процессов:
\begin{itemize}
\item kernel namespaces -- пространства имен позволяют изолировать пользователей, сетевые интерфейсы, межпроцессное взаимодействие и файловую систему;
\item AppArmor и SELinux для мандатного разграничения доступа;
\item seccomp политики -- это фильтры безопасного вычисления, которые позволяют разрешать или запрещать системные вызовы для контейнера;
\item chroots вместе с pivot\_root перемещает корневую файловую систему в подготовленный образ операционной системы;
\item kernel capabilities -- разрешения ядра позволяют выдавать контейнеру только необходимые разрешения на использование возможностей ядра без предоставления прав суперпользователя;
\item cgroups -- контрольные группы позволяют выставлять квоты на использование аппаратных ресурсов и управлять доступом к устройствам.
\end{itemize}
\vspace{\baselineskip}

\begin{center}
\textbf{ЛИТЕРАТУРА:}
\end{center}

\begin{enumerate}
\item What is container technology? [Электронный ресурс]. – Режим доступа: https://www.techradar.com/news/what-is-container-technology (дата обращения: 8.11.2018).
\item Linux-контейнеры: изоляция как технологический прорыв [Электронный ресурс]. – Режим доступа: https://habr.com/company/redhatrussia/blog/352052/ (дата обращения: 8.11.2018).
\item LXD crushes KVM in density and speed [Электронный ресурс] - Режим доступа: https://blog.ubuntu.com/2015/05/18/lxd-crushes-kvm-in-density-and-speed (дата обращения: 10.11.2018).
\end{enumerate}
\clearpage